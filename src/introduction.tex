\IEEEPARstart{S}{ince} the invention of the first television, the display system has manifested into a critical human-computer-interface 
capable of bringing vivid and abundant visual information to users. 
As technology evolved, flat panel LCD display systems were invented and are now 
widely used in various multimedia systems, ranging from large screen home theatre systems 
to personal laptops and mobile devices. As projected in ~\cite{lcd_future}, LCD TVs are becoming progressively bigger
in size every year, with customers demanding powerful visual experiences as video technology evolves. As the smartphone industry
explores newer dimensions of innovation, there is an increasing demand for larger screens in this multimedia category too~\cite{iPhone}.

The LCD panel in a common LCD display system will not emit light by itself. Therefore, a backlight panel and a light-diffusing 
component are used under the LCD panel as a source of lighting in the system.
Currently, there are two kinds of lighting sources employed for the backlight panel. 
An older method uses cold cathode fluorescent lamps (CCFL); while a more recent one utilizes 
light-emitting diodes (LEDs) since LEDs can offer greater dynamic contrast, wider color gamut and 
less power consumption. Although this new backlight technology can provide better power efficiency, 
it needs to be pointed out that the backlight panel still remains the largest portion of the entire 
system power consumption ~\cite{Sinofsky}. With the increasing demand for larger screens, 
optimizing and enhancing the power efficiency of the LCD display system, 
especially televisions ~\cite{Samsung}, is thus attracting continuing research efforts.

While approximate computing is becoming a powerful paradigm to save energy in the vision space~\cite{tvlsi2015}, 
another promising way to solve the power problem specific to LCD panels is to dim the LED backlight. Many different methods 
have been proposed in the same vein. Active dimming approaches adjust the luminance level of the backlight 
based on pixel information such as contrast, color, or brightness. Passive dimming methods modify the 
luminance level by monitoring user attention with a camera or sensors. 
The backlight is dimmed when users are away and restored to full level when users are in front of the display.

In ~\cite{DATE2013}, an active dimming strategy is proposed, where visual saliency 
is used to adaptively change the luminance level of the backlight panel at a zone granularity. Most objects of interest
have distinct features that stand out in contrast to their background. 
~\cite{DATE2013} used three feature channels - \emph{intensity}, \emph{color} and \emph{orientation} - for locating salient or 
interesting objects/regions in the scene. These low-level features, when applied to images, 
perform extremely well, as was demonstrated in ~\cite{DATE2013}. However, due to the static nature of these 
channels, the system, when operating on a video stream, is constrained in that it fails to behave 
like a person who can focus attention on new objects entering the frame, or moving objects across a series of frames. 
To overcome these handicaps, we extend ~\cite{DATE2013} with the following contributions:
\begin{itemize}
\item
LCD-based devices these days invariably have streaming video applications running on them and consume more power than when a static image is being observed. 
As highlighted in ~\cite{Peters2007}, two additional processing channels, \emph{motion} and \emph{flicker} are effective in finding salient regions 
in a temporal environment. We thus incorporate these two channels into ~\cite{DATE2013} to account for dynamic occurrences 
across a stream of video frames.
\item
We then highlight the impact of these changes on user experience. 
Since the original system settings in ~\cite{DATE2013} introduce a \emph{shimmer} (discussed later) in the final compensated video, 
new dimming and compensation coefficients are used. A movement constraint for salient zones between two 
continuous frames is also adopted for preserving video quality.
Using these new compensation coefficients, verification of our system is undertaken with 29 different videos as part of a user perception evaluation test.
\item 
To address system constraints such as power, performance and resource tradeoffs,
this paper adopts a generic field-programmable gate array (FPGA) channel architecture. 
On average, a \textbf{50\%} power saving can be achieved with our extended system while maintaining real-time constraints. 
\end{itemize}
The organization of the rest of this paper is as follows: Section~\ref{sec:related} highlights related work on bio-inspired systems in general and various LED power saving schemes in specific; Section~\ref{sec:saliency} introduces Saliency, the adopted biological attention model, in detail; the LED backlight system and its power model is outlined in Section~\ref{sec:led}; Section~\ref{sec:sysarch} shows the design and implementation details of the proposed power management system; all the experimental results are provided in Section~\ref{sec:results}; finally we conclude the paper in Section~\ref{sec:conclusion}.


Imagine you find yourself at home and your pantry is missing something that you want. How do you fix this? You go to your local grocery store, walk around a bit, peruse some other interesting products, pick the product you want at the store. 

