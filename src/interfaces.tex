Our visual assistive system consists of several interacting components: 
\begin{itemize}
\item One of the devices that we use is a pair of smart glasses. The smart
glasses have both a camera and a built in headset, along with
networking capability. These built in glasses provide to our system a
point of view of the person wearing them which is crucial when need to
guide someone to a product. Also because they are worn like normal
sunglasses, they aren't intrusive which is something that was really
important to the users who tested our system.

\item We use a specially designed prototype glove that has been
modified to have both a camera attached to it, as well set of
vibration motors. This camera that is on the glove allows the system
to have the view point of where the person is reaching. This view
point may be different from that of ...

\item EDGE COMPUTING (shopping cart / tablet / etc -- need to make consistent across all tasks in introduction)

\item CLOUD AND SERVER COMPONENTS

\end{itemize}

///Talk about other sensors


\subsection{Interfaces}
One of the most important aspects of any technology is how a human
interacts with it. Having an intuitive, simple, and functional
interface can often be the difference between a successful device and
one which isn't~\cite{iPhone}. This importance on a interface is even more important
when trying to help someone with an impairment. Not only does the
interface have to be all the things mentioned above, but it also has
to be robust for different environments. In the case of assisting a
person with visual impairment this means being able to handle cases
which people without visual impairment handle without even realizing
they do.

For our assistive technology we employ two main modes of providing
this feedback and guidance to the user. these modes are auditory
feedback and tactile feedback. 



//Talk about how the system is used in practice: glove vibrates, directions, etc.
//talk about glasses? tablet?
