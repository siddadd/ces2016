<<<<<<< HEAD
~\cite{something}
One of the most important aspects of any technology is how a human interacts with it. Having an intuitive, simple, and functional interface can often be the difference between a successful device and one which isn't. This importance on a interface is even more important when trying to help someone with an impairment. Not only does the interface have to be all the things mentioned above, but it also has to be robust for different environments. In the case of assisting a person with visual impairment this means being able to handle cases which people without visual impairment handle without even realizing they do. Such a scenario would be when reaching for a product if you momentarily move your head in a different direction.
=======
Our visual assistive system consists of several interacting components: 
\begin{itemize}
\item One of the devices that we use is a pair of smart glasses. The smart
glasses have both a camera and a built in headset, along with
networking capability. These built in glasses provide to our system a
point of view of the person wearing them which is crucial when need to
guide someone to a product. Also because they are worn like normal
sunglasses, they aren't intrusive which is something that was really
important to the users who tested our system.
>>>>>>> c815808eb514ff65e5a5151aee6268c04c79d321

\item We use a specially designed prototype glove that has been
modified to have both a camera attached to it, as well set of
vibration motors. This camera that is on the glove allows the system
to have the view point of where the person is reaching. This view
point may be different from that of ...

<<<<<<< HEAD
One of the devices that we use is a pair of smart glasses. The smart glasses have both a camera and a built in headset, along with networking capability. These built in glasses provide to our system a point of view of the person wearing them which is crucial when need to guide someone to a product. Also because they are worn like normal sunglasses, they aren't intrusive which is something that was really important to the users who tested our system. The headset on the "smart" glasses allows for the delivery of the auditory feedback.
One of the devies that we use is a prototype glove that has been modified to have both a camera attached to it, as well set of vibration motors. This camera that is on the glove allows the system to have the view point of where the person is reaching. The vibration motors attached to the glove allow use to provide subtle feedback to the user the convey to them which when they would have to move their hand to be able to grab the desired product. An example of this would be buzzing the right motor to indicate a rightward motion.
=======
\item EDGE COMPUTING (shopping cart / tablet / etc -- need to make consistent across all tasks in introduction)
>>>>>>> c815808eb514ff65e5a5151aee6268c04c79d321

\item CLOUD AND SERVER COMPONENTS

<<<<<<< HEAD

=======
\end{itemize}
>>>>>>> c815808eb514ff65e5a5151aee6268c04c79d321

Use of Other Sensors:
    Indoor GPS
        can be used for positioning to reduce to computation required for localization
    IMU / Magnomter:
        Used for diffing the pose of hand versus the head helping to address the challenge of fine grain control.


Challenges.
    Alignment:
        While 

    User movement speed:
        With any system that provides user feedback, being able to keep up with the speed at which the user moves. In a system that is meant to provide assistance and guidance the fluidity of the interface is crucial. The slightest delay in this type of system could lead to a variety of situations such as the user walking into something, missing the intended product, or picking the wrong product. In a task that requires image processing this problem is even more difficult due to the computation requriement of the backend.



<<<<<<< HEAD
=======

\subsection{Interfaces}
One of the most important aspects of any technology is how a human
interacts with it. Having an intuitive, simple, and functional
interface can often be the difference between a successful device and
one which isn't~\cite{iPhone}. This importance on a interface is even more important
when trying to help someone with an impairment. Not only does the
interface have to be all the things mentioned above, but it also has
to be robust for different environments. In the case of assisting a
person with visual impairment this means being able to handle cases
which people without visual impairment handle without even realizing
they do.

For our assistive technology we employ two main modes of providing
this feedback and guidance to the user. these modes are auditory
feedback and tactile feedback.  ... MORE ...

\subsection{Using the System}
While certain aspects of system use differ across the particular tasks
it supports, the modes and mechanisms employed during grocery shopping
provide good coverage of typical operations, and we describe them in
detail below.

//Talk about how the system is used in practice: glove vibrates, directions, etc.
//talk about glasses? tablet?
>>>>>>> c815808eb514ff65e5a5151aee6268c04c79d321
